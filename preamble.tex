% ---------------------
% Pacotes OBRIGATÓRIOS
% ---------------------
\usepackage{lastpage}			% Usado pela Ficha catalográfica
\usepackage{indentfirst}			% Indenta o primeiro parágrafo de cada seção.
\usepackage{xcolor}			     	% Controle das cores
\usepackage{graphicx}	         % Inclusão de gráficos
\usepackage{epsfig,subfig}		% Inclusão de figuras
\usepackage{microtype} 			% Melhorias de justificação
% ---------------------
		
% ---------------------
% Pacotes ADICIONAIS
% ---------------------
\usepackage{lipsum}						% Geração de dummy text
\usepackage{amsmath,amssymb,mathrsfs, amsthm}	% Comandos matemáticos avançados 
\usepackage{setspace}  					% Para permitir espaçamento simples, 1 1/2 e duplo
\usepackage{verbatim}					% Para poder usar o ambiente "comment"
\usepackage{tabularx} 					% Para poder ter tabelas com colunas de largura auto-ajustável
\usepackage{afterpage} 					% Para executar um comando depois do fim da página corrente
\usepackage{url} 						% Para formatar URLs (endereços da Web)
\usepackage{todonotes}  		% Lista de afazeres To-dos
\usepackage{enumitem}					% Fazer enumerações por letras ou números nos itens
\usepackage{float}							% Controla a posição de figuras e tabelas
\usepackage{longtable}					% Tabelas que se estendem por mais de uma página
\usepackage{booktabs}					% Tabelas com multicolunas
\usepackage{makecell}					% Notas de rodapé da tabela com várias linhas
\usepackage{csquotes}
% ---------------------

% ---------------------
% Pacotes de CITAÇÕES
% ---------------------
%\usepackage[brazilian,hyperpageref]{backref}	% Paginas com as citações na bibl
%\usepackage[alf]{abntex2cite}				% Citações padrão ABNT (alfa)
%\usepackage[num]{abntex2cite}				% Citações padrão ABNT (numericas)
% ---------------------

% Configurações de ambiente de teoremas e definições
% --- 
% Configura o ambiente de teoremas, definições etc.
% --- 


\theoremstyle{definition} 
\newtheorem{teor}{Teorema}[chapter]
\newtheorem{defi}[teor]{Definição}
\newtheorem{lema}[teor]{Lema}
\newtheorem{supo}[teor]{Suposição}
\newtheorem{exemplo}[teor]{Exemplo}
\newtheorem{prop}[teor]{Proposição}


% Inclusão de dados para CAPA e FOLHA DE ROSTO (título, autor, orientador, etc.)
% ---
% Informações de dados para CAPA e FOLHA DE ROSTO
% ---
%\titulo{Cálculo de Medidas de Risco de Mercado sob a Perspectiva do Acordo de Basileia III}
%\autor{Rafael Felipe Bressan}
\local{Florianópolis - SC}
%\data{Xxxx de 20XX}
\orientador{Prof. Dr. Daniel Augusto de Souza}
%\coorientador{Fulano Coorientador}
\instituicao{%
  Universidade do Estado de Santa Catarina - UDESC
  \par
  Centro de Ciências da Administração e Sócio Econômicas - Esag
  \par
  Departamento de Ciências Econômicas}
\tipotrabalho{Monografia}
% O preambulo deve conter o tipo do trabalho, o objetivo,
% o nome da instituição e a área de concentração
\preambulo{\textbf{Monografia} apresentada ao Departamento de Ciências Econômicas da Universidade do Estado de Santa Catarina, como requisito parcial para obtenção do título de Bacharel em Ciências Econômicas, orientado pelo Prof. Dr. Daniel Augusto de Souza.}
% ---

% Inclui Configurações de aparência do PDF Final
%  Configurações de aparência do PDF final
% NÃO ALTERAR!!!

% alterando o aspecto da cor azul
\definecolor{blue}{RGB}{41,5,195}

% informações do PDF
\makeatletter
\hypersetup{
     	%pagebackref=true,
		pdftitle={\@title}, 
		pdfauthor={\@author},
    		pdfsubject={\imprimirpreambulo},
	    pdfcreator={LaTeX with abnTeX2},
		pdfkeywords={abnt}{latex}{abntex}{abntex2}{trabalho acadêmico}, 
		colorlinks=true,       		% false: boxed links; true: colored links
    		linkcolor=blue,          	% color of internal links
    		citecolor=blue,        		% color of links to bibliography
    		filecolor=magenta,      		% color of file links
		urlcolor=blue,
		bookmarksdepth=4
} 
\makeatother
% --- 

% Controle do espaçamento entre um parágrafo e outro:
\setlength{\parskip}{1.5\onelineskip}  % tente também \onelineskip
% O tamanho da identação do parágrafo é dado por:
\setlength{\parindent}{1.25cm}
\makeatletter
\def\thm@space@setup{%
  \thm@preskip=8pt plus 2pt minus 4pt
  \thm@postskip=\thm@preskip
}
\makeatother

% A identação e margens de itemizações deve ser igual ao parágrafo
\setlength{\leftmargini}{1.7cm}

% ---------------------
% Compila o indice
% ---------------------
\makeindex
% ---------------------
